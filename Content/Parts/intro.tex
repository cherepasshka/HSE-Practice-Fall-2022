%!TEX root = ../main

\section{Введение}


Графические процессоры способны быстро обрабатывать большие объемы данных, но имеют меньшую точность по сравнению с центральными процессорами, но тем не менее точности, достигаемой с помощью GPU, хватает для решения многих задач, в том числе для большинства задач машинного обучения. В ML очень важна скорость обучения моделей, особенно она критична для громоздких моделей с большим количеством параметров, например, для нейронных сетей, поэтому возникает резонное желание ускорить процесс тренировки моделей за счет распределённого обучения и использования графических процессоров. 

В рамках практики мы изучили несколько статей на эту тему и собрали их разборы в этом отчете.
